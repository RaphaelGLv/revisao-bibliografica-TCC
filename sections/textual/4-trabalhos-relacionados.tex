\section{Trabalhos relacionados}

A gestão da variedade de dados, um desafio inerente a aplicações modernas e sistemas de big data, tem sido objeto de extensiva pesquisa e diversas abordagens arquitetônicas.
Diversas abordagens foram propostas para lidar com este problema.

\subsection{O fim dos SGBDs generalistas e a ascensão da persistência poliglota}
Inicialmente, os SGBDs (principalmente os relacionais) eram usados de forma generalista, centralizando todos os dados, independentemente do formato, em um único banco de dados.

Essa abordagem, apesar de extremamente popular entre os anos 80 e 2000, se mostrou ineficiente para lidar com a variedade de dados, como discutido por \textcite{StonebrakerAndCetintemel2005}, que criticaram a ideia dominante chamada por eles de \textit{``one size fits all''} (``um tamanho serve para todos'', tradução livre) como algo que já não fazia mais sentido.

Neste mesmo artigo, os autores argumentam que a solução para o problema da variedade de dados seria o uso de múltiplos SGBDs especializados, cada um lidando com um tipo específico de dado.

Essa solução, posteriormente nomeada como persistência poliglota \cite{Leberknight2011}, é amplamente adotada atualmente, com diversas soluções especializadas para diferentes tipos de dados, continuando a usar SGBDs relacionais para dados estruturados, mas integrando bancos de dados NoSQL para dados não estruturados, bancos de dados em grafos para dados relacionais complexos, entre outros.
\begin{figure}[H]
    \centering
    \caption{Exemplo de arquitetura de persistência poliglota}
    \includegraphics[width=0.5\linewidth]{images/Fowler2011-polyglot-persistence.png}\\
    \fonteimagem{Fonte: \textcite{Fowler2011}}
    \label{fig:Fowler2011-polyglot-persistence}
\end{figure}

\subsection{Persistência poliglota e suas características}
A mudança causada pela substancial influência de \textcite{StonebrakerAndCetintemel2005} é discutida por \textcite{Lajam2022}, que classifica os diferentes SGBDs com base em seus modelos de dados, define e categoriza as abordagens de persistência poliglota, além de discutir suas implicações práticas.

\subsubsection{Abordagens de persistência poliglota}
\textcite{Lajam2022} realiza uma profunda análise das diferentes implementações de persistência poliglota, categorizando-as em quatro tipos:
\begin{itemize}
    \item \textbf{Coordenada pela aplicação:} A lógica da aplicação é responsável por interagir com múltiplos SGBDs, decidindo qual banco de dados usar para cada operação.
    \item \textbf{Orientada a serviços: } Cada serviço dentro de uma arquitetura de microsserviços é responsável por gerenciar seu próprio SGBD, comunicando-se com outros serviços conforme necessário.
    \item \textbf{Como um serviço:} Existe uma camada intermediária (middleware) que serve de fachada para o acesso aos múltiplos bancos de dados.
    \item \textbf{Bancos de dados multi-modelo:} Utiliza SGBDs que suportam múltiplos modelos de dados, permitindo armazenar diferentes tipos de dados dentro do mesmo sistema, internalizando a lógica e complexidade da persistência poliglota em um único SGBD.
\end{itemize}

\subsubsection{Desafios da persistência poliglota}
Anterior a \textcite{Lajam2022}, \textcite{Fowler2011} previu, precisamente, que o uso da persistência poliglota seria uma tendência crescente entre as empresas de maior porte. Mas, apesar de reconhecer os benefícios, \textcite{Fowler2011}, junto com \textcite{Lajam2022}, também destaca os desafios e complexidades adicionais que essa abordagem traz, sendo elas:
\begin{itemize}
    \item \textbf{Complexidade de gerenciamento:} A introdução de múltiplos SGBDs aumenta a complexidade do sistema, exigindo habilidades especializadas para gerenciar e manter cada banco de dados.
    \item \textbf{Integração de dados:} A integração de dados entre diferentes SGBDs pode ser desafiadora, especialmente quando os dados precisam ser combinados ou sincronizados.
    \item \textbf{Consistência e disponibilidade de dados:} Garantir a consistência e disponibilidade dos dados entre diferentes SGBDs pode ser complexo, especialmente em sistemas distribuídos.
\end{itemize}

\subsection{Bancos de dados multi-modelo}

O conceito de bancos de dados multi-modelo é abordado e resumido por \textcite{PluciennikAndZgorzatek2017}, que, apesar de constatarem que o conceito não ter uma definição formal bem estabelecida, definem esse conceito como a capacidade de um SGBD de suportar múltiplos formatos de dados (como relacional, documentos, grafos, etc.) em uma única plataforma. Permitindo, assim, que diferentes tipos de dados sejam armazenados e gerenciados no modelo apropriado para sua estrutura e seu processamento.

Dessa forma, os bancos de dados multi-modelo buscam oferecer a flexibilidade e a eficiência da persistência poliglota, mas com a simplicidade de gerenciar um único sistema, mitigando alguns dos desafios associados à abordagem tradicional de persistência poliglota. Dessa forma, a abordagem multimodelo emerge como uma tentativa de equilibrar a simplicidade de gestão dos antigos sistemas generalistas com a flexibilidade da persistência poliglota, buscando mitigar a complexidade de gerenciamento e a integração de dados inerente de um sistema heterogêneo de bancos de dados.

\textcite{PluciennikAndZgorzatek2017} acrescentam, após uma comparação simplificada entre alguns exemplos de bancos de dados multi-modelo, que a adoção dessa abordagem, apesar de apresentar latência ligeiramente maior que o banco de dados especializado analisado, é promissora, visto que suporta a maioria dos modelos que julga necessários e fornece maior flexibilidade e habilidade de lidar com buscas inter-modelo de forma unificada.
\begin{figure}[H]
    \centering
    \caption{Resultados comparativos entre bancos de dados multi-modelo (ArangoDB, OrientDB e Couchbase) e banco de dados especializado (MongoDB)}
    \includegraphics[width=0.5\linewidth]{images/PluciennikAndZgorzatek2017-multimodel-dbs-comparison.png}\\
    \fonteimagem{Fonte: \textcite{PluciennikAndZgorzatek2017}}
    \label{fig:PluciennikAndZgorzatek2017-multimodel-dbs-comparison}
\end{figure}

Contudo, \textcite{PluciennikAndZgorzatek2017} não concluem que os bancos de dados multi-modelo sejam a solução definitiva para o problema da variedade de dados, mas sim uma alternativa viável que merece consideração e análises mais aprofundadas de testes e índices da adesão e popularidade dessas tecnologias.

\subsection{Comparações e benchmarks}
A comparação entre bancos de dados é uma tarefa comum, porém extremamente complexa e subestimada, o que resulta em comparações imprecisas, enviesadas, e muitas vezes inconclusivas.

Para abordar essa questão, \textcite{Taipalus2024} realizou uma revisão sistemática da literatura, analisando artigos de cinco grandes bases de dados acadêmicas (IEEE Xplore, ACM Digital Library, SpringerLink, ScienceDirect e Google Scholar), selecionando todos artigos entre 2010 e 2021 que compararam o desempenho de pelo menos dois SGBDs, o que resultou em 51 artigos para serem analisados.

A partir desses artigos, \textcite{Taipalus2024} extraiu dados sobre os SGBDs comparados, os tipos de cargas de trabalho usadas, as métricas de desempenho avaliadas, e as ferramentas e metodologias empregadas, obtendo três resultados principais:
\begin{itemize}
    \item \textbf{Métricas comuns:} As métricas de desempenho mais utilizadas nos estudos são Latência (tempo de resposta) e Throughput (vazão/operações por segundo). Métricas de consumo de recursos como uso de CPU e memória também são comuns, mas menos exploradas.
    \item \textbf{Desempenho relativo:} Não existe um SGBD universalmente melhor. O desempenho depende da carga de trabalho e do modelo de dados. Em geral, bancos NoSQL (por exemplo MongoDB e Cassandra) tendem a oferecer melhor desempenho e escalabilidade para dados não estruturados e altas cargas de escrita, enquanto bancos SQL se destacam em consistência (ACID) e consultas complexas.
    \item \textbf{Lacunas na literatura:} Muitos estudos carecem de rigor metodológico. Demonstrando ausência de uma metodologia bem definida, uso de cargas de trabalho inconsistentes e falta de detalhes para garantir a reprodutibilidade dos experimentos.
\end{itemize}

A partir desses resultados, \textcite{Taipalus2024} conclui que a escolha do SGBD deve ser baseada nas necessidades específicas da aplicação, considerando o tipo de dados, o padrão de acesso e os requisitos de desempenho. Além disso, destaca a necessidade de estudos mais rigorosos e padronizados para superar as lacunas identificadas na literatura.

Observando essas mesmas dificuldades e as substanciais diferenças entre bancos de dados, \textcite{ZhangAndLu2021} propuseram uma metodologia de comparação entre bancos de dados dedicada para casos em que há uma grande variedade de dados (\textit{``UniBench''}).

Para isso, os autores definiram um conjunto de dados, métricas e cargas de trabalho para simular do mundo real, assim fornecendo uma base consistente e reprodutível para comparar o desempenho de diferentes SGBDs.    