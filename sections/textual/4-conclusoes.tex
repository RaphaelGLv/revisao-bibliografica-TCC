\section{Conclusões}

Lidar com um conjunto de dados heterogêneo é um desafio antigo, porém cada vez mais relevante no cenário atual. Foi diante deste contexto que o NoSQL e, principalmente, a persistência poliglota ascenderam como soluções viáveis para esse problema. Contudo, essa abordagem traz consigo uma complexidade adicional que torna a gestão e manutenção do sistema mais desafiadora. Nesse sentido, os bancos de dados multi-modelo surgem como uma alternativa promissora, buscando equilibrar a flexibilidade da persistência poliglota com a simplicidade de gerenciar um único sistema.

Para avaliar essas abordagens, a literatura destaca a importância de comparações rigorosas e padronizadas entre SGBDs, como evidenciado pela revisão sistemática de \textcite{Taipalus2024} e pela metodologia proposta por \textcite{ZhangAndLu2021}. Portanto, este trabalho visa contribuir para essa discussão, analisando, na próxima etapa, as duas abordagens principais para o gerenciamento de dados heterogêneos: sistemas heterogêneos de bancos de dados e bancos de dados multi-modelo.

Para investigar empiricamente este trade-off, este trabalho propõe um caso de estudo prático: o desenvolvimento de uma rede social. Esta aplicação é ideal para este propósito, pois seus requisitos (perfis de usuário, postagens, conexões de amizade) espelham a heterogeneidade de dados discutida. O objetivo é, portanto, aplicar uma metodologia de benchmark rigorosa, inspirada nos princípios do UniBench, para comparar as duas arquiteturas. A análise buscará obter dados quantitativos sobre desempenho e escalabilidade (Latência e Throughput) e qualitativos sobre a complexidade de implementação, contribuindo com evidências práticas para o debate acadêmico.