\section{Introdução}

Ao estarmos inseridos em meio a quarta revolução industrial, caracterizada pela expressiva gama de dados gerados diariamente, bem como a alta conexão entre dispositivos, sistemas e pessoas, a complexidade da gestão de dados torna-se cada vez mais desafiadora. Ainda mais considerando que as aplicações mais utilizadas globalmente são uma lista de, majoritariamente, redes sociais, como TikTok, Instagram e Facebook\footnote{Disponível em: \url{https://www.statista.com/statistics/1285960/top-downloaded-mobile-apps-worldwide/}. Acesso em: 10 jun. 2024.}, que lidam com uma variedade imensa de dados, desde perfis de usuários, postagens, comentários, diversos tipos de conexões entre usuários, entre outros.

Essa alta diversidade de dados em um único contexto é conhecida como heterogeneidade de dados. Essa heterogeneidade, somada com o volume exorbitante de dados gerados diariamente e a alta velocidade e disponibilidade exigida por aplicações modernas, impõe desafios significativos para o gerenciamento eficiente e eficaz desses dados. Sendo assim, qual a melhor abordagem para lidar com essa variedade de dados?

Este desafio já foi e ainda é abordado de diferentes formas. Inicialmente, com o grande sucesso dos Sistemas de Gerenciamento de Bancos de Dados (SGBDs) relacionais, a abordagem predominante era centralizar todos os dados em um único banco de dados relacional. Posteriormente, com o surgimento do movimento NoSQL (\textit{Not Only SQL}), novas abordagens foram propostas, como o uso de múltiplos SGBDs especializados para diferentes tipos de dados, uma abordagem conhecida como persistência poliglota \cite{Leberknight2011}. Por fim, mais recentemente, os bancos de dados multi-modelo surgem como uma alternativa promissora, buscando equilibrar a flexibilidade da persistência poliglota com a simplicidade de gerenciar um único sistema \cite{PluciennikAndZgorzatek2017}.

Com a existência dessas diferentes abordagens, surge a necessidade de compará-las de forma rigorosa e padronizada, a fim de avaliar seu desempenho e eficácia em cenários reais, tomando conta dos pontos fortes e fracos de cada uma. Nesse sentido, este trabalho busca contribuir para este tema, não só analisando os estudos já existentes na literatura, mas também propondo um estudo de caso prático, observando justamente o trade-off entre sistemas heterogêneos de bancos de dados e bancos de dados multi-modelo em um sistema com alta heterogeneidade de dados e alta popularidade: uma rede social.