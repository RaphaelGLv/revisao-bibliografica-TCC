% Resumo/palavras-chave (língua vernácula)------------------------------------------------------
\newpage
\thispagestyle{empty}
\begin{center}
    \uppercase{\textbf{Resumo}}
\end{center}
Nos últimos anos, o volume e a variedade de dados gerados e armazenados por aplicações de software aumentaram significativamente. Com a proliferação de dispositivos conectados, redes sociais, serviços em nuvem e outras fontes de dados, as aplicações modernas frequentemente lidam com uma ampla gama de tipos e formatos de dados, desde dados estruturados até dados não estruturados e semi-estruturados. Essa diversidade de dados apresenta desafios significativos para o gerenciamento eficiente e eficaz desses dados, exigindo abordagens inovadoras e flexíveis.

Este desafio é solucionado atualmente a partir de uma abordagem conhecida como persistência poliglota, mas, devido à sua complexidade, surge uma nova alternativa: os bancos de dados multi-modelo. Esses sistemas se destacam como uma solução promissora, buscando equilibrar a flexibilidade da persistência poliglota com a simplicidade de gerenciar um único sistema.

Diante disso, este trabalho tem como objetivo analisar e comparar as duas principais abordagens para o gerenciamento de dados heterogêneos: sistemas heterogêneos de bancos de dados e bancos de dados multi-modelo. Através de uma revisão sistemática da literatura, serão explorados os conceitos fundamentais, as vantagens e desvantagens de cada abordagem, bem como as metodologias utilizadas para avaliar seu desempenho e eficácia.

\begin{center}
    \uppercase{\textbf{Palavras-chaves}}
\end{center}
Dados heterogêneos, Persistência poliglota, Bancos de dados multi-modelo, Gerenciamento de dados, Arquiteturas de dados.
