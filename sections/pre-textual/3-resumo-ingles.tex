\newpage
\thispagestyle{empty}
\begin{center}
    \textbf{ABSTRACT}
\end{center}
In the past years, the volume and variety of data generated and stored by software applications have increased significantly. With the proliferation of connected devices, social networks, cloud services, and other data sources, modern applications often deal with a wide range of data types and formats, from structured data to unstructured and semi-structured data. This data diversity presents significant challenges for the efficient and effective management of this data, requiring innovative and flexible approaches.

This challenge is currently addressed through an approach known as polyglot persistence, but due to its complexity, a new alternative emerges: multi-model databases. These systems stand out as a promising solution, seeking to balance the flexibility of polyglot persistence with the simplicity of managing a single system.

In this context, this paper aims to analyze and compare the two main approaches to managing heterogeneous data: heterogeneous database systems and multi-model databases. Through a systematic literature review, the fundamental concepts, advantages, and disadvantages of each approach will be explored, as well as the methodologies used to evaluate their performance and effectiveness.
\begin{center}
    \textbf{KEYWORDS}
\end{center}
Heterogeneous data, Polyglot persistence, Multi-model databases, Data management, Data architectures.
